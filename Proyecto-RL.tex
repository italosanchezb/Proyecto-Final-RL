% Options for packages loaded elsewhere
\PassOptionsToPackage{unicode}{hyperref}
\PassOptionsToPackage{hyphens}{url}
\PassOptionsToPackage{dvipsnames,svgnames,x11names}{xcolor}
%
\documentclass[
  us-letterpaper,
  DIV=11,
  numbers=noendperiod]{scrreprt}

\usepackage{amsmath,amssymb}
\usepackage{iftex}
\ifPDFTeX
  \usepackage[T1]{fontenc}
  \usepackage[utf8]{inputenc}
  \usepackage{textcomp} % provide euro and other symbols
\else % if luatex or xetex
  \usepackage{unicode-math}
  \defaultfontfeatures{Scale=MatchLowercase}
  \defaultfontfeatures[\rmfamily]{Ligatures=TeX,Scale=1}
\fi
\usepackage{lmodern}
\ifPDFTeX\else  
    % xetex/luatex font selection
\fi
% Use upquote if available, for straight quotes in verbatim environments
\IfFileExists{upquote.sty}{\usepackage{upquote}}{}
\IfFileExists{microtype.sty}{% use microtype if available
  \usepackage[]{microtype}
  \UseMicrotypeSet[protrusion]{basicmath} % disable protrusion for tt fonts
}{}
\makeatletter
\@ifundefined{KOMAClassName}{% if non-KOMA class
  \IfFileExists{parskip.sty}{%
    \usepackage{parskip}
  }{% else
    \setlength{\parindent}{0pt}
    \setlength{\parskip}{6pt plus 2pt minus 1pt}}
}{% if KOMA class
  \KOMAoptions{parskip=half}}
\makeatother
\usepackage{xcolor}
\setlength{\emergencystretch}{3em} % prevent overfull lines
\setcounter{secnumdepth}{5}
% Make \paragraph and \subparagraph free-standing
\makeatletter
\ifx\paragraph\undefined\else
  \let\oldparagraph\paragraph
  \renewcommand{\paragraph}{
    \@ifstar
      \xxxParagraphStar
      \xxxParagraphNoStar
  }
  \newcommand{\xxxParagraphStar}[1]{\oldparagraph*{#1}\mbox{}}
  \newcommand{\xxxParagraphNoStar}[1]{\oldparagraph{#1}\mbox{}}
\fi
\ifx\subparagraph\undefined\else
  \let\oldsubparagraph\subparagraph
  \renewcommand{\subparagraph}{
    \@ifstar
      \xxxSubParagraphStar
      \xxxSubParagraphNoStar
  }
  \newcommand{\xxxSubParagraphStar}[1]{\oldsubparagraph*{#1}\mbox{}}
  \newcommand{\xxxSubParagraphNoStar}[1]{\oldsubparagraph{#1}\mbox{}}
\fi
\makeatother


\providecommand{\tightlist}{%
  \setlength{\itemsep}{0pt}\setlength{\parskip}{0pt}}\usepackage{longtable,booktabs,array}
\usepackage{calc} % for calculating minipage widths
% Correct order of tables after \paragraph or \subparagraph
\usepackage{etoolbox}
\makeatletter
\patchcmd\longtable{\par}{\if@noskipsec\mbox{}\fi\par}{}{}
\makeatother
% Allow footnotes in longtable head/foot
\IfFileExists{footnotehyper.sty}{\usepackage{footnotehyper}}{\usepackage{footnote}}
\makesavenoteenv{longtable}
\usepackage{graphicx}
\makeatletter
\def\maxwidth{\ifdim\Gin@nat@width>\linewidth\linewidth\else\Gin@nat@width\fi}
\def\maxheight{\ifdim\Gin@nat@height>\textheight\textheight\else\Gin@nat@height\fi}
\makeatother
% Scale images if necessary, so that they will not overflow the page
% margins by default, and it is still possible to overwrite the defaults
% using explicit options in \includegraphics[width, height, ...]{}
\setkeys{Gin}{width=\maxwidth,height=\maxheight,keepaspectratio}
% Set default figure placement to htbp
\makeatletter
\def\fps@figure{htbp}
\makeatother
% definitions for citeproc citations
\NewDocumentCommand\citeproctext{}{}
\NewDocumentCommand\citeproc{mm}{%
  \begingroup\def\citeproctext{#2}\cite{#1}\endgroup}
\makeatletter
 % allow citations to break across lines
 \let\@cite@ofmt\@firstofone
 % avoid brackets around text for \cite:
 \def\@biblabel#1{}
 \def\@cite#1#2{{#1\if@tempswa , #2\fi}}
\makeatother
\newlength{\cslhangindent}
\setlength{\cslhangindent}{1.5em}
\newlength{\csllabelwidth}
\setlength{\csllabelwidth}{3em}
\newenvironment{CSLReferences}[2] % #1 hanging-indent, #2 entry-spacing
 {\begin{list}{}{%
  \setlength{\itemindent}{0pt}
  \setlength{\leftmargin}{0pt}
  \setlength{\parsep}{0pt}
  % turn on hanging indent if param 1 is 1
  \ifodd #1
   \setlength{\leftmargin}{\cslhangindent}
   \setlength{\itemindent}{-1\cslhangindent}
  \fi
  % set entry spacing
  \setlength{\itemsep}{#2\baselineskip}}}
 {\end{list}}
\usepackage{calc}
\newcommand{\CSLBlock}[1]{\hfill\break\parbox[t]{\linewidth}{\strut\ignorespaces#1\strut}}
\newcommand{\CSLLeftMargin}[1]{\parbox[t]{\csllabelwidth}{\strut#1\strut}}
\newcommand{\CSLRightInline}[1]{\parbox[t]{\linewidth - \csllabelwidth}{\strut#1\strut}}
\newcommand{\CSLIndent}[1]{\hspace{\cslhangindent}#1}


\KOMAoption{captions}{tableheading}
\makeatletter
\@ifpackageloaded{bookmark}{}{\usepackage{bookmark}}
\makeatother
\makeatletter
\@ifpackageloaded{caption}{}{\usepackage{caption}}
\AtBeginDocument{%
\ifdefined\contentsname
  \renewcommand*\contentsname{Tabla de contenidos}
\else
  \newcommand\contentsname{Tabla de contenidos}
\fi
\ifdefined\listfigurename
  \renewcommand*\listfigurename{Listado de Figuras}
\else
  \newcommand\listfigurename{Listado de Figuras}
\fi
\ifdefined\listtablename
  \renewcommand*\listtablename{Listado de Tablas}
\else
  \newcommand\listtablename{Listado de Tablas}
\fi
\ifdefined\figurename
  \renewcommand*\figurename{Figura}
\else
  \newcommand\figurename{Figura}
\fi
\ifdefined\tablename
  \renewcommand*\tablename{Tabla}
\else
  \newcommand\tablename{Tabla}
\fi
}
\@ifpackageloaded{float}{}{\usepackage{float}}
\floatstyle{ruled}
\@ifundefined{c@chapter}{\newfloat{codelisting}{h}{lop}}{\newfloat{codelisting}{h}{lop}[chapter]}
\floatname{codelisting}{Listado}
\newcommand*\listoflistings{\listof{codelisting}{Listado de Listados}}
\makeatother
\makeatletter
\makeatother
\makeatletter
\@ifpackageloaded{caption}{}{\usepackage{caption}}
\@ifpackageloaded{subcaption}{}{\usepackage{subcaption}}
\makeatother

\ifLuaTeX
\usepackage[bidi=basic]{babel}
\else
\usepackage[bidi=default]{babel}
\fi
\babelprovide[main,import]{spanish}
% get rid of language-specific shorthands (see #6817):
\let\LanguageShortHands\languageshorthands
\def\languageshorthands#1{}
\ifLuaTeX
  \usepackage{selnolig}  % disable illegal ligatures
\fi
\usepackage{bookmark}

\IfFileExists{xurl.sty}{\usepackage{xurl}}{} % add URL line breaks if available
\urlstyle{same} % disable monospaced font for URLs
\hypersetup{
  pdftitle={Análisis Futbol},
  pdflang={es-MX},
  colorlinks=true,
  linkcolor={blue},
  filecolor={Maroon},
  citecolor={Blue},
  urlcolor={Blue},
  pdfcreator={LaTeX via pandoc}}


\title{Análisis Futbol}
\author{}
\date{2024-02-14}

\begin{document}
\maketitle

\renewcommand*\contentsname{Tabla de contenidos}
{
\hypersetup{linkcolor=}
\setcounter{tocdepth}{2}
\tableofcontents
}

\bookmarksetup{startatroot}

\chapter*{Preface}\label{preface}
\addcontentsline{toc}{chapter}{Preface}

\markboth{Preface}{Preface}

\bookmarksetup{startatroot}

\chapter{Introduction}\label{introduction}

This is a book created from markdown and executable code.

See Van Roy et~al. (2021) for additional discussion of literate
programming.

\bookmarksetup{startatroot}

\chapter{Formulación}\label{formulaciuxf3n}

El Proceso de Decisión de Markov se compone de los siguientes elementos:

\begin{itemize}
\item
  El conjunto de estados estará conformado por las 3 divisiones del
  campo \(c_1,c_2,c_3\), además se agregan tres estados absorbentes:

  \begin{itemize}
  \tightlist
  \item
    \(lp =\) pérdida de posesión del balón.
  \item
    \(ng =\) realizar un tiro y que no termine en gol.
  \item
    \(g =\) realizar un tiro y que termine en gol. De esta forma el
    conjunto de estados \(\mathcal{S}\) queda como
  \end{itemize}

  \[
    \mathcal{S}=\{c_1,c_2,c_3,lp,ng,g\}
    \]
\item
  El conjunto de acciones admisibles se considerarán 3 acciones que
  serán

  \begin{itemize}
  \tightlist
  \item
    \(t\) = tiro
  \item
    \(p\) = pase
  \item
    \(r\) = regate
  \end{itemize}

  De esta forma el conjunto de acciones queda como
  \[\mathcal{A} = \{t,p,r\}\]
\item
  Para las de transiciones haremos uso de las probabilidades de
  transición definidas de la siguiente forma: \[
   P:\mathcal{S}\times\mathcal{A}\times\mathcal{S} \to [0,1]
   \]

  Que se interpreta como la probabilidad de estar en un estado \(s_i\)
  realizar una acción \(a_k\) y terminar en un estado \(s_j\). Notemos
  que se aceptan los casos cuando \(i=j\) y más adelante se mostrará que
  algunas probabilidades serán 0.
\item
  La función de recompensa
  \(R:\mathcal{S}\times \mathcal{A}\times \mathcal{S}\to[0,1]\) será \[
    R(s_i,a_k,s_j)=\left\{\begin{array}{ccc}
                 1 & \text{si} & s_j=g  \\
                 0 & o.c. &
            \end{array}\right.
    \]
\end{itemize}

\bookmarksetup{startatroot}

\chapter{Dinámica del modelo}\label{dinuxe1mica-del-modelo}

En el contexto del fútbol llamamos \emph{jugada} a una sucesión de
acciones donde el balón se traslada desde un punto inicial donde el
equipo A tiene el balón hasta un punto final que puede ser: perder el
balón, tirar a puerta y no anotar gol o tirar y anotar gol.

\emph{Ejemplo: El balón comienza en el saque de meta del portero, el
portero da un pase a un defensa que se encuentra en el primer tercio,
que esté da un pase a un delantero que se encuentra en el tercer tercio
y al intentar un regate pierde el balón.}

En nuestro contexto se verá como el hecho de iniciar la sucesión de
acciones desde alguna sección \(c_i\) y terminar en alguno de los 3
estados absorbentes. \emph{Ejemplo} \[
c_1\xrightarrow{p}c_1\xrightarrow{p}c_3\xrightarrow{r}lp.
\]

Para movernos de un estado \(s_i\) a un estado \(s_j\) mediante una
acción \(a_k\) haremos uso de las probabilidades de transición, estas
probabilidades las estimaremos utilizando datos extraídos de FBREF para
4 clubes particulares: Chivas, América, Cruz Azul y Pumas.

\bookmarksetup{startatroot}

\chapter{Descripción y Justificación de la
recompensa}\label{descripciuxf3n-y-justificaciuxf3n-de-la-recompensa}

En un partido de fútbol gana el equipo que anota más goles, en caso de
anotar los mismos goles se considera empate y no existe desempate de
ningún tipo.\footnote{No se consideran los partidos de eliminación
  directa donde existe el desempate por penales.} Por lo que la
recompensa será la de anotar un gol \(R=1\), pues es lo único que puede
hacer que un equipo gane un partido. No existe penalización porque los
goles válidos anotados no pueden ser descontandos.

\bookmarksetup{startatroot}

\chapter{Justificación de las
acciones}\label{justificaciuxf3n-de-las-acciones}

Las acciones que puede realizar un equipo durante un partido son
limitadas y se pueden enlistar. Sin embargo para nuestro modelo vamos a
seleccionar las 3 más importantes que son el \emph{pase, tiro}y
\emph{regate}.

\begin{itemize}
\tightlist
\item
  \emph{tiro}: Es la acción que nos permite anotar goles.
\item
  \emph{pase}: Ayuda a un equipo a mover el balón por el campo sin
  necesidad de desplazarse o dejar rivales atrás.
\item
  \emph{regate}: Permite que podamos trasladar el balón de un lugar a
  otro y dejar a rivales atrás.
\end{itemize}

\bookmarksetup{startatroot}

\chapter*{References}\label{references}
\addcontentsline{toc}{chapter}{References}

\markboth{References}{References}

\phantomsection\label{refs}
\begin{CSLReferences}{1}{0}
\bibitem[\citeproctext]{ref-van2021analyzing}
Van Roy, Maaike, Wen-Chi Yang, Luc De Raedt, y Jesse Davis. 2021.
{``Analyzing learned markov decision processes using model checking for
providing tactical advice in professional soccer''}. En \emph{AI for
Sports Analytics (AISA) Workshop at IJCAI 2021}.

\end{CSLReferences}




\end{document}
